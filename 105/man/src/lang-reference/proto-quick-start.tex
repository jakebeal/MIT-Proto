\documentclass{article}
\usepackage{fullpage}
\usepackage{graphicx}
\usepackage{../svn-multi}
\svnid{$Id: proto-quick-start.tex 10 2008-08-28 14:05:10Z jakebeal $}

\title{Proto Quick Start}
\author{Jacob Beal}
\date{Date: \today, SVN Version: \svnrev{}}

\newcommand\todo[1]{\immediate\write16{TODO: #1}}

\newcommand\violation{{\em Capable of violating the continuous
    space/time abstraction.}}

% proto code
\newcommand\code[1]{\begin{quote}\var{#1}\end{quote}}
\newcommand\function[2]
{\begin{quote}{\tt #1}: #2 \end{quote}}
\newcommand\type[1]{$#1$}
\newcommand\var[1]{{\tt #1}}

% arguments/keystrokes
\newcommand\key[1]{{\bf #1}}
\newcommand\simarg[2]{\begin{quote} {\bf Argument: \var{#1}} \\ #2 \end{quote}}
\newcommand\simkey[2]{\begin{quote} {\bf Key: \key{#1}} \\ #2 \end{quote}}
\newcommand\simmouse[2]{\begin{quote} {\bf Mouse: \key{#1}} \\ #2 \end{quote}}
\newcommand\simargkey[3]{
  \begin{quote} {\bf Argument: \var{#1}, Key: \key{#2}} \\ #3 \end{quote}
}
\newcommand\simPMarg[3]{
  \begin{quote}
    {\bf Positive Argument: \var{#1}, Negative Argument: \var{#2}} \\ #3
  \end{quote}
}
\newcommand\simPMargkey[4]{
  \begin{quote}
    {\bf Positive Argument: \var{#1}, Negative Argument: \var{#2}, 
      Key: \key{#3}} \\ #4
  \end{quote}
}
\newcommand\true{{\bf TRUE}}
\newcommand\false{{\bf FALSE}}


\begin{document}

\maketitle

This is a quick-start guide for Proto, giving a quick tour of commonly
used features of the language and simulator.  For installation
instructions, see the {\bf Proto Installation Guide}.  For a tutorial
on the Proto language, see the document {\bf Thinking In Proto}.  For
a reference to the Proto language, see the {\bf Proto Language
  Reference}.  For a usee manual for the simulator, see the {\bf Proto
  Simulator User Manual}.  For information on how to extend the
functionality of the simulator, see the {\bf Proto Simulator Developer
  Reference}.

% standard LaTeX credits insert; should echo AUTHORS

\section{Credits for Proto}

The Proto language was created in partnership by Jonathan Bachrach and
Jacob Beal.  As they created the language, Jonathan created the first
implementation of MIT Proto, including the first compiler, kernel,
simulator, and embedded device implementations.  Since that time, Jake
and other contributors have built on the work begun by Jonathan.

MIT Proto also includes contributions from (alphabetically):
%
Aaron Adler, Geoffrey Bays, Anna Derbakova, Nelson Elhage, Takeshi
Fujiwara, Tony Grue, Joshua Horowitz, Tom Hsu, Kanak Kshetri, Prakash
Manghwani, Dustin Mitchell, Omar Mysore, Maciej Pacula, Hayes Raffle,
Dany Qumsiyeh, Omari Stephens, Mark Tobenkin, Ray Tomlinson, Kyle
Usbeck, Dan Vickery

The Protobo platform code in platforms/protobo/ also includes 
Topobo-related code from (alphabetically):
%
  Mike Fleder, Limor Fried, Josh Lifton, Laura Yip


\section{Smoke Test}

Let us assume you have already successfully built Proto.
Go to the directory \var{resrc/} and type:
\code{./proto -n 1000 -r 10 -l -c -T -v "(red (gradient (once (< (rnd
  0 1) 0.01))))"}
You should see a green network with blue unreadable text at the
intersections.  Red dots will spread through the screen from
several starting locations as a magenta number in the lower
left counts upward.

Drag with your left mouse button to rotate the display.  You will see
that the red dots form a mountainous landscape.  Drag with your right
mouse button to zoom and get a better view.

Let's break down this command: \var{-n 1000} means use 1000 devices
and \var{-r 10} means connect devices within 10 meters of one another.
The \var{-c} means show the network: it's the green thing; hit 'c' to
turn it off.  You'll notice the magenta number in the lower right go
up when you do that: it's displaying how many frames per second the
simulator is rendering, and the one in the lower left is displaying
the number of simulated seconds that have elapsed.  The timing display
was invoked by \var{-T}; hit 'T' (shift-t) to turn them off.

Now you are looking at just the blue numbers and the red dots, and
it's time to delve into the Proto expression a little: it's the big
thing inside the quotes.  The expression \var{(rnd 0 1)} means that
all 1000 devices should pick random floating point numbers between 0
and 1.  The comparison \var{(< (rnd 0 1) 0.01)} turns these into a
field of boolean numbers which is true at approximately 10 devices.
The \var{once} function wrapped around this says to do this once and
remember the result.  Then we feed this boolean field into the
\var{gradient} function, which finds the distance from each device to
the nearest device with a true value---that is, the nearest one that
picked a random number less than 0.01.  Finally, these distances are
fed to the \var{red} LED actuator that produces those dots on the
screen.  If you zoom in one the blue numbers, you'll see that the dots
float above the number at a height equal to the number.  The blue
numbers are the output of the Proto expression, and are visible
because of the \var{-v}: you can hit 'n' to turn them on and off.  The
LEDs are visible because of the \var{-l} and you can use 'L' to turn
them on and off.

\section{Proto Language Essentials}

Here are some essentials to the Proto language, along with the
most frequently used functions.

\paragraph{Evaluation}

Proto is a purely functional language.  Proto is written using
s-expressions in a manner very similar to Scheme.  Evaluating a Proto
expression produces a program: a dataflow graph that may be evaluated
against a space to produce an evolving field of values at every point
on the space.

\paragraph{Data Types}

All Proto expressions produce fields that map every point in space to
a value.  The values produced are categorized into four basic types:
fields, lambdas, tuples, and scalars.  A number is a scalar or vector
(tuple with scalar values), a local is anything but a field, and a
boolean is a scalar interpreted as a logical value: false is 0,
anything else is true.

\paragraph{Namespaces and Bindings}

Proto is a lexically scoped language.  Names are not case sensitive.
Bindings contain values and are looked up by name.  Lexical bindings
are visible only within the scope in which they are bound, and shadow
bindings of the same name from enclosing scopes.

When the Proto compiler encounters an unknown identifier $name$, it
searches its path for a file named {\tt $name$.proto}.  If it finds
such a file, then it loads the contents of the file and looks up the
identifier again.  Definitions in subdirectories can be accessed with
identifiers of the form {\tt $dir$/$name$}.

\function{(def .name (.arg ...) ,@body)}{Define a function \var{name}
  in the current scope, with as many arguments as there are \var{arg}
  identifiers.  The body is evaluated within an extended scope where
  the \var{arg} identifiers are bound to arguments to the function.
  \var{fun}, which omits \var{name}, creates anonymous functions.}

\function{(let ((.var ,value) ...) ,@body)}{Extends scope,
  binding all \var{var} identifiers to their associated \var{value} 
  in parallel.  The \var{body} is evaluated in the extended scope.
  \var{let*} is like let, except that identifies are bound sequentially,
  so later ones can use earlier ones in their definition.}

\paragraph{Control Flow}

\function{(all ,@forms)}{All \var{forms} are evaluated in parallel
  and the value of the last form returned.}

\function{(mux ,test ,true ,false)}{Evaluates both
  \var{true} and \var{false} expressions.  When \var{test} is true,
  returns the result of the \var{true} expression, otherwise returns
  the result of the \var{false} expression.  The \var{true} and
  \var{false} expressions must return the same type.}

\function{(if ,test ,true ,false)}{Restricts execution to
  subspaces based on \var{test}.  Where \var{test} is true, the
  \var{true} expression is evaluated; where \var{test} is false, the
  \var{false} expression is evaluated.  The \var{true} and \var{false}
  expressions must return the same type.}

\paragraph{State}

Because Proto is a purely functional language, we create state using
feedback loops.  A state variable is initialized at some value, then
evolves that value forward in time.  In regions where the feedback
loop is not evaluated, the state variable is reinitialized, resuming
evolution when the feedback loop begins to be evaluated again.

For example, the expression: \code{(rep t 0 (+ t (dt)))}
creates a timer that returns how long evaluation has been proceeding
at each device.

\function{(letfed ((.var ,init ,evolve) ...)  ,@body)}{Creates a state
  variable for each \var{var}.  \var{var} is initially bound to the
  value of expression \var{init}, and at each time step the state is
  evolved forward using expression \var{evolve}.  The body is
  evaluated within an extended scope including the state variables.
  
  In the \var{evolve} expression, each \var{var} is bound to an old
  value and \var{(dt)} is set to the time since the last step.
  All \var{init} and \var{evolve} expressions are evaluated in
  parallel, so no variable can reference another value in its
  \var{init}, but variables can use one another's old values in their
  \var{evolve} statements.  \violation{}.}

\function{(rep .var ,init ,evolve)}{Create a single feedback variable
  and return its value.  Equivalent to \var{(letfed ((.var ,init
    ,evolve)) .var)}.  \violation{}}

\function{(once ,expr)}{Evaluates \var{expr} once, then always returns
  that value.}

\paragraph{Logic and Arithmetic}

\begin{itemize}
\item Logical operators: \var{and}, \var{or}, \var{not}
\item Constants: \var{(inf)}, \var{(e)}, \var{(pi)}
\item Arithmetic: \var{+}, \var{-}, \var{*},  \var{/}, \var{neg} (negation)
\item Comparison: \var{=}, \var{<}, \var{>}, \var{<=}, \var{>=}
\item Functions: \var{pow}, \var{min}, \var{max}, \var{sqrt}, \var{abs}, 
  \var{sin}, \var{cos}, \var{atan2}
\item Random Numbers: \var{(rnd ,min ,max)} gives a floating point
  random number in the range $[min, max]$.
\item Vectors: \var{(vdot ,a ,b)}, \var{(normalize ,v)}, 
  \var{(polar-to-rect ,v)}, \var{(rect-to-polar ,v)}
\end{itemize}

\paragraph{Tuples}

\function{(tuple ,v ++)}{Creates a tuple with the set of
  \var{v} arguments as its elements.}

\function{(elt ,tuple ,i)}{Returns the \var{i}th element of
  \var{tuple}, counting from zero.  \var{1st}, \var{2nd}, and
  \var{3rd} are functions for getting elements 0, 1, and 2.}

\paragraph{Neighborhoods}

There are two types of neighborhood functions: functions that create
fields, and functions that summarize fields into local values.  In
between, any pointwise function can be applied to fields, producing a
field whose values are the result of applying the pointwise operation
to the values of the input fields.

\begin{itemize}
\item \var{(nbr ,expr)}: Returns a field mapping neighbors to their
  values of \var{expr}.
\item \var{(nbr-range)}, \var{(nbr-angle)}, \var{(nbr-lag)}, \var{(nbr-vec)}:
  return a field of distances, bearings, time lags, and vectors to
  neighbors, respectively.
\item \var{min-hood}, \var{max-hood}, \var{all-hood}, \var{any-hood},
  \var{int-hood}: summarize a field into a scalar that is,
  respectively, minimum, maximum, for-all, existence, and integral
  over the values of the field.
\end{itemize}

\function{(fold-hood ,fold ,base ,value)}{Collects \var{value} from
  each of the neighbors, then folds these into a summary value, using
  \var{fold} to combine elements into \var{base} one at a
  time. \violation{}}

\paragraph{Sensor and Actuators}

Actuators reset themselves to a null value whenever they are not
actively being invoked.  Thus, for example, 
\code{(if (sense 1) (mov (tup 2)) (red (tup 1)))} 
will cause devices move to the right only when \var{(sense 1)} is
true, and to turn on their red LED only when \var{(sense 1)} is false.

\function{(mov ,velocity)}{Attempt to move at \var{velocity}.  The
  return echoes \var{velocity}.}

\function{(red ,n)}{Set red LED to intensity \var{n}.  Intensity
  ranges from 0 to 1, but overloading of display can show values outside
  this range.  The return echoes \var{n}.  \var{green} and \var{blue}
  are identical, but set the green and blue LEDs instead.}

\function{(probe ,value ,i)}{Posts \var{value} to the \var{i}th probe
  (valid indices are 0 to 2).}

\function{(sense ,i)}{Returns the \var{i}th user sensor value.}

\function{(clone ,now)}{When \var{now} is true, the device attempts to
  reproduce.  The return echoes \var{now}.}

\function{(die ,now)}{When \var{now} is true, the device attempts to
  suicide.  The return echoes \var{now}.}

\function{(coord)}{Returns the device's estimated coordinates.}
\function{(hood-radius)}{Returns the maximum expected range at which
  devices can communicate.}

\paragraph{Library Functions}

These are not primitive functions, but are frequently used building
blocks which have been included in Proto's distribution library, in
the directory {\tt lib/}.

\function{(distance-to ,source)}{Calculates the shortest-path
  distance from every device to the set of devices where \var{source}
  is true.  The function \var{gradient} is an alias.}

\function{(broadcast ,source ,value)}{Flow \var{value} outward from
  devices in the \var{source} to all other devices.  Each device takes
  its value from the nearest \var{source} device.}

\function{(dilate ,source ,d)}{Returns true for every device within
  distance \var{d} of the \var{source}.}

\function{(distance ,region1 ,region2)}{Calculates the distance
  between \var{region1} and \var{region2} and broadcasts it
  everywhere.}

\function{(disperse)}{Devices repel from one another using spring
  forces.}

\function{(dither)}{Devices wander randomly in a 2D plane.}

\function{(elect)}{Devices choose a leader by mutual exclusion
  and maintain precisely one leader within a given distance.}

\function{(timer)}{Return the length that this device has been
  evaluating this expression (i.e. not going in different branches of
  an \var{if})}.

\section{Simulator Essentials}

\simkey{q}{Quit the simulator.}
\simargkey{-v}{n}{Show value computed at each device.  Toggled by key.}
\simargkey{-sv}{v}{When device outputs a 2- or 3-tuple, display it as
  a vector (toggled by key).}

\simargkey{-T}{T}{Display simulator time in lower left corner and
  frames-per-second in lower right corner. Toggled by key.}
\simargkey{-step}{s}{Use stepping mode, advancing one step on key
  \key{s}} 
\simkey{x}{Execute freely (ending stepping mode).}
\simarg{-s N}{Set simulated seconds per step, default
  0.01/\var{ratio}}

\paragraph{Display:}
\simargkey{-f}{f}{Full screen display (toggled by key)}
\todo{Need to add GLUT geometry arguments}
\simkey{z}{Reset display to initial view.}
\simkey{ARROW KEYS}{Shift simulation display in the direction of the arrow, 
  in the simulation's coordinate system.}
\simmouse{LEFT DRAG}{Rotate display.}
\simmouse{RIGHT DRAG}{Zoom display; toward center zooms out, away zooms in.}
  
\paragraph{Selection:}
\simmouse{LEFT CLICK}{Select the devices clicked on.}
\simmouse{RIGHT CLICK}{Select, then print the state of the selected
  device(s) to standard out.}
\simmouse{SHIFT LEFT DRAG}{Toggle the selection of all devices in a
  rectangular display region.}
\simmouse{SHIFT RIGHT DRAG}{Move selected devices.}

\paragraph{Device Distribution}
Simulations are created with \var{n} devices distributed through a
bounded 2D or 3D volume according to one of several distribution
rules.

\simarg{-n N}{Number of devices.}
\simarg{-3d}{Use a 3D distribution; default is 2D.}
\simarg{-grid}{Distributes devices in a grid, rathen than uniformly
  randomly.}

\paragraph{Unit Disc Radio Communication}
\simarg{-r N}{Transmission range for radio, default 15.}
\simarg{-ns N}{Set transmission range to get an expected neighborhood
  size of \var{N}.  Overrides \var{-r}.}
\simargkey{-c}{c}{Display network connections (toggled by key).}

\paragraph{Dynamics}
There are two dynamics packages: ``simple dynamics,'' which does
not handle collisions and allows instantaneous shifts in velocty, 
and ``ODE dynamics,'' which is based on the Open Dynamics Engine,
a Newtonian physics simulator.  Simple dynamics is the default.

\simarg{-ODE}{Use ODE dynamics for physics.}
\simarg{-rad N}{Radius of a device body, defaults from the initial
  distribution bounds: $\sqrt{0.087*(width*height/n)}$}
\simargkey{-w}{w}{Use walls to keep devices inside the initial distribution
  bounds.}
\simargkey{-m}{m}{Enable movement (toggled by key).}

\paragraph{Sensors and Actuators}
\simargkey{-probes N}{p}{Display the first \var{N} of the three probes.  
  The \var{p} key cycles through how many are shown.}
\simargkey{-l}{L}{Display LEDs (toggled by key).}
\simkey{t}{Toggle user sensor 1 on selected devices.}
\simkey{y}{Toggle user sensor 2 on selected devices.}
\simkey{u}{Toggle user sensor 3 on selected devices.}
\simkey{B}{Clone selected devices.}
\simkey{K}{Kill selected devices.}

\end{document}
